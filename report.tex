\documentclass[12pt,a4paper]{jsarticle}

\usepackage[utf8]{inputenc}
\usepackage[T1]{fontenc}
\usepackage[dvipdfmx]{graphicx}
\usepackage{lmodern}
\usepackage{amsmath,amssymb}
\usepackage{hyperref, url}
\usepackage{float}

\title{情報理工学域 メディア情報学プログラム\\
    プログラミング演習 最終レポート}
\author{DX2 グループ11\\
    2411593 齋藤寛仁\\
    2411603 佐々木晃誠\\
    2411712 Puller Stefano Daniele}
\date{\today}

\begin{document}
\maketitle

\newpage
\section{概要説明}
どの様な目的のプログラムを実現したか,どのような機能を実装したか,をまず最初に説明してください.必要に応じてスクリーンキャプチャした画像も張り付けてください.
「ドローエディタ」ならば,どのような機能を持ったドローエディタを作ったか説明してください.
作業の分担など,作業をどのように進めたかも説明して下さい.

\section{設計方針}
プログラムの基本構想を述べて下さい.
どのようなアルゴリズム,データ構造を用いて,目的の処理を実現するのか説明 しましょう.なぜ,そのようなアルゴリズム,データ構造を用いたかという考察 も含めてください.ただし,ここでは,プログラムコードの細部には触れる必要 はありません.それは次の章で行います.
今回はJavaプログラミングなので,どのようなクラスを用意して, どのように利用するかを説明してください. クラス図を使用して,主要なクラス間の関係を説明するしましょう.
なお,クラス図を書くときは, すでにあるものを利用したクラス(レポートで特に説明しないクラス)と, 新しく作成したクラスは区別が付くように しましょう.

\section{プログラムの説明}
作成したすべてのクラスを説明してください. 主な各クラスの主なメソッド,フィールドの説明を行ってください. 必要に応じてプログラムリストを抜粋して説明しましょう.
ここは, 必ず,メンバ全員がそれぞれ自分の担当した部分についての 説明を書くようにしてください. 発表会でのプレゼンと同様に 各自の担当部分のセールスポイト(実装した機能,採用したデータ構造やアルゴリズムなどについて) も書いて下さい. 「文責:○○」 も忘れずに書くようにしてください.
本授業HPにあるひな形コード,もしくはWeb上のコードを ひな形として利用した場合は,その旨明記して下さい. Web上のコードを元にした場合は,必ずURLも書いてください。その場合,ひな型に最初から含まれていた,主要なメソッド,フィールド についても説明を行って,※などの印を付けて,追加したメソッド,フィールド と区別できるようにしてください.

\section{実行例}
実行例をスクリーンショットを使って, 分かりやすく説明してください. 特に,作成したプログラムの特徴となるような機能に ついては,実際の使用例を分かりやすく説明するようにしてください.

\section{考察}
完成したプログラムに対する考察,コメントを述べてください. 当初予定していた通りの物ができたか考察してください. また,考察を踏まえて,今後の改良点についても述べてください.

\section{感想}
グループでの作業を通しての反省や感想,それから考察できること.
今後の各自の担当部分の課題.やり残したこと.
Javaやオブジェクト指向,MVCモデルなど学習内容に関する感想。
「プログラミング演習」の授業に関する感想や要望.

\section{付録1:操作法マニュアル}
そのプログラムを初めて使う人向けの説明書を書いてください. ドローエディタなら操作法の説明,ゲームならキー操作の 説明などを書いてください.スクリーンキャプチャした画像に 説明を書き込んだりしてもいいでしょう. 2ページ程度の簡潔な説明書で構いません。

\section{付録2:プログラムリスト}
プログラムリストは本文中には説明に必要な部分だけ抜粋して掲載して, プログラムリスト全体はレポートの最後に「付録」として付ける. プログラムコードには,主なメソッド,フィールドの説明など 最低限のコメントは付ける。さらに,cat -n コマンドなどで行番号を付与して から載せること.

\end{document}