\documentclass[12pt,a4paper]{jsarticle}

\usepackage[utf8]{inputenc}
\usepackage[T1]{fontenc}
\usepackage[dvipdfmx]{graphicx}
\usepackage{lmodern}
\usepackage{amsmath,amssymb}
\usepackage{hyperref, url}
\usepackage{caption}
\usepackage{float}
\usepackage{here}
\usepackage{array}
\usepackage{listings, xcolor}

% --- Java用の色定義 ---
\definecolor{gray}{rgb}{0.5,0.5,0.5}

% --- 共通設定 ---
\lstset{
    language=Java,                % 言語をJavaに設定
    basicstyle={\ttfamily\small}, % 標準のフォント
    keywordstyle=\color{red},     % キーワード(public, class等)の色
    commentstyle=\color{gray},    % コメントの色
    stringstyle=\color{blue},     % 文字列の色
    numbers=left,                 % 行番号を左に表示
    numberstyle=\tiny\color{gray},% 行番号のサイズと色
    stepnumber=1,                 % 行番号のステップ
    numbersep=10pt,               % 行番号とコードの距離
    backgroundcolor=\color{white},% 背景色
    showspaces=false,             % スペースを記号で表示しない
    showstringspaces=false,       % 文字列中のスペースを記号で表示しない
    showtabs=false,               % タブを記号で表示しない
    frame=single,                 % 外枠(1本線)
    rulecolor=\color{black},      % 枠線の色
    tabsize=4,                    % タブの幅
    captionpos=t,                 % キャプションを上に配置
    breaklines=true,              % 長い行を自動折り返し
    breakatwhitespace=false,      % スペース以外でも折り返す
    title=\lstname,               % ファイル名を表示
    columns=[l]{fullflexible},    % 文字の間隔(日本語対応)
    keepspaces=true               % 空白を維持
}

\title{情報理工学域 メディア情報学プログラム\\
    プログラミング演習 最終レポート}
\author{DX2 グループ11 Gladiator
    \vspace{4mm}\\
    2411593 齋藤寛仁\\
    2411603 佐々木晃誠\\
    2411712 Puller Stefano Daniele}
\date{2026年2月18日}

\begin{document}
\maketitle

\newpage
\section{概要説明}
% どの様な目的のプログラムを実現したか,どのような機能を実装したか,をまず最初に説明してください.
% 必要に応じてスクリーンキャプチャした画像も張り付けてください.
% 作業の分担など,作業をどのように進めたかも説明して下さい.

私たちのグループは、シンプルなシューティングゲームを作成しました。画面上部から敵がランダムに降りてきて攻撃します。敵を躱したり、矢を放って敵を迎え撃ちます。敵を倒すことでスコアが上昇し、一定のスコアに達するとボスが登場します。各ステージのボスを倒すことで、次のステージに移動します。ボスを倒すとそのキャラの特殊スキルを自分の能力として使うことができるようになります。プレイヤーのライフが0になるとゲームオーバーです。\\
\indent 基本的なキー操作は移動と攻撃で、カーソルキーで移動し、Spaceキーで矢を放って攻撃します。プレイ中に一時停止することも可能です。\\
\indent プレイヤーのライフは3が上限で、敵に当たるか敵の攻撃を受けると1減ります。ボスが登場したタイミングと、ボスを倒して次のステージに進んだタイミングでライフが全回復します。\\
\indent プレイ画面の上部には、現在のスコアとステージ数、プレイヤーの残りライフが表示されています。特殊スキルは画面下部に順次追加されます。
\begin{figure}[H]
    \centering
    \includegraphics[scale=0.25]{play.eps}
    \caption{プレイ画面}
\end{figure}
MVCモデルを採用し、主にPullerがM、佐々木がV、斎藤がCを担当しました。プログラムのコードはGitHubで管理しました。また、「実装予定機能リスト」を作成し、未開発の機能を赤、開発中の機能をオレンジ、開発済みの機能を緑にし、開発中の機能の最後に『(名前)』を書いて、誰が今どの機能を開発中なのかを一目で把握できるようにしました。

{\raggedleft (文責:佐々木)\par}

\section{設計方針}
% プログラムの基本構想を述べて下さい.
% どのようなアルゴリズム,データ構造を用いて,目的の処理を実現するのか説明しましょう.なぜそのようなアルゴリズム,データ構造を用いたかという考察も含めてください.
% ただし,ここでは,プログラムコードの細部には触れる必要はありません.それは次の章で行います.
% 今回はJavaプログラミングなので,どのようなクラスを用意して,どのように利用するかを説明してください。クラス図を使用して,主要なクラス間の関係を説明するしましょう。
% なお,クラス図を書くときは,すでにあるものを利用したクラス(レポートで特に説明しないクラス)と,新しく作成したクラスは区別が付くようにしましょう.
\subsection{基本構想}
本プログラムは、JavaのSwingライブラリを用いた縦スクロール型の2Dシューティングゲームである。プレイヤーは自機を操作し、ランダムに出現する敵やボスを撃破しながらステージを進行する。開発にあたっては、拡張性と保守性を高めるため、適切なデザインパターンとデータ構造を選定した。

\subsection{採用したアルゴリズムとデータ構造}
目的の処理を実現するために、以下のデータ構造とアルゴリズムを採用した。

\begin{itemize}
    \item \textbf{MVCモデルの採用 (アーキテクチャ)}\\
    プログラムの責務を明確に分離するため、Model-View-Controllerアーキテクチャを採用した。
    \begin{itemize}
        \item \textbf{理由}: ゲームロジック(当たり判定や敵の出現管理)と描画処理が混在すると、コードが肥大化しデバッグが困難になるためである。Modelがデータを保持し、Viewが描画に専念することで、複数人での並行開発(ロジック担当とUI担当の分業)を円滑に進めることができた。
    \end{itemize}

    \item \textbf{ArrayListによる動的オブジェクト管理 (データ構造)}\\
    ゲーム内に登場する敵キャラクターや弾丸は、ゲーム進行に伴い頻繁に生成・消滅を繰り返す。そのため、固定長配列ではなく、可変長配列である\texttt{java.util.ArrayList}を採用した。
    \begin{itemize}
        \item \textbf{理由}: 画面上のオブジェクト数は予測不可能であるため、要素の追加・削除が容易なリスト構造が必要であった。メインループ内ではこのリストを走査し、一括して移動と描画を行っている。
    \end{itemize}

    \item \textbf{Areaクラスによる精密な当たり判定 (アルゴリズム)}\\
    矩形判定(\texttt{Rectangle})に加え、より精密な判定が必要な場合に\texttt{java.awt.geom.Area}クラスを用いた領域計算アルゴリズムを採用した。
    \begin{itemize}
        \item \textbf{理由}: 単純な四角形同士の判定では、円形の弾丸や不規則な形状の敵に対して「当たっていないのに当たった」という違和感が生じるため、画像の不透明部分に基づいた正確な衝突判定を実現するためである。
    \end{itemize}
\end{itemize}

\subsection{クラス構成とクラス図}
効率的な実装を行うため、オブジェクト指向の継承とポリモーフィズムを積極的に活用した。主要なクラス構成を以下に示す。

\begin{itemize}
    \item \textbf{GameObjectクラス (基底クラス)}: 全てのキャラクターの親クラス。座標$(x, y)$、画像、および抽象メソッド\texttt{move()}, \texttt{draw()}を持つ。これにより、異なる種類のオブジェクトを同一のリストで管理可能にしている。
    \item \textbf{HostileEntityクラス}: \texttt{GameObject}を継承し、敵キャラクター共通の機能(HP管理、被ダメージ処理、スコア加算)を実装している。
    \item \textbf{Bossクラス / Minionクラス}: \texttt{HostileEntity}をさらに具体化し、ボス特有の振る舞いや雑魚敵の挙動を定義している。
\end{itemize}

クラス間の関係図を図\ref{fig:class_diagram}に示す。本図に含まれるクラスは全て本プロジェクトのために新規に設計・実装したものである。また、図中の色分けは各クラスの実装担当者を表しており、MVCモデルに基づいた分担が行われていることを示している。

\begin{figure}[H]
    \centering
    \includegraphics[width=1.0\columnwidth]{class_diagram.eps}
    \caption{クラス構成図と開発担当}
    \label{fig:class_diagram}
\end{figure}


\section{プログラムの説明}
% 作成したすべてのクラスを説明してください.主な各クラスの主なメソッド,フィールドの説明を行ってください.必要に応じてプログラムリストを抜粋して説明しましょう.
% ここは,必ず,メンバー全員がそれぞれ自分の担当した部分についての説明を書くようにしてください.発表会でのプレゼンと同様に各自の担当部分のセールスポイト(実装した機能,採用したデータ構造やアルゴリズムなどについて) も書いて下さい.「文責:○○」も忘れずに書くようにしてください.
% 本授業HPにあるひな形コード,もしくはWeb上のコードをひな形として利用した場合は,その旨明記して下さい。Web上のコードを元にした場合は,必ずURLも書いてください。その場合,ひな型に最初から含まれていた,主要なメソッド,フィールドについても説明を行って,※などの印を付けて,追加したメソッド,フィールドと区別できるようにしてください。

\subsection{齋藤担当}
私が作成したのは、コントローラーと第二ボスZeusの実装です。

\subsubsection{コントローラーの実装}
\lstinputlisting[
  caption={GamePanel.java(一部抜粋)},
  firstline=382,
  lastline=409
]{src/view/GamePanel.java}
キー操作は、使用するキーが押されている状態か押されていない状態かをtrue,falseで表して処理しています。updatePlayerVelocityでは、変数vx,vyを用いて、キーが押された方向にPlayerが上下左右に動くことができるようにした。例えば、左のカーソルキーだけ押されている状態であれば、Playerは左に移動することとなり、座標としては、負の方向に進むはずなのでvx=-1となるようにした。
キー操作の入力を検知してから、vx,vyを変更するという操作をすると、方向転換の時に、止まってしまってうまく方向転換ができなかった。しかし、true,falseの判別によりそれを解消することができ、より滑らかなキー操作が可能になった。

続いて、resetKeyStateです。ここは、ゲームを再開したり、ゲームオーバーになってゲームを新しく始めるときなどの状態を初期化するときに、用いるプログラムです。ここは、上で説明したプログラムで変更した状態を元の状態に変更します。元の状態に変更しないと再開と同時にPlayerが動き始めてしまうので、方向キーの状態は、全てfalseにし、変数vx,vyも0に戻します。

\lstinputlisting[
  caption={GamePanel.java(一部抜粋)},
  firstline=411,
  lastline=485
]{src/view/GamePanel.java}
今までは、Playerの移動の操作をしました。keyPressedでは、移動以外の処理を行なっています。ここでは、Gameの状態に合わせて条件分岐処理を行っています。

ゲームのタイトルが出ている時は、スペースキーを押してスタートするような仕様にしているので、7,8行目のように、ゲームの状態がTITLEの時に、スペースキーを押されたら、ゲーム開始するようにしています。

続いて、16行目から47行目です。ゲームをしている時は、Playerの移動だけではなく、矢を放ったり、一時停止や、特殊能力まであり、ここもこれら全て条件分岐で処理をしています。まず、移動は、キーが押されたときの処理を行っているため、押されている方向キーのところは全てtrueに変更しています。そして、方向キーの押されているかの処理を終えた後、Playerが動かないといけないので、updatePlayerVelocityを実行し、方向キーの変更を移動という形で表せるようにした。他の動作については、動作にあったキーを押された時に、その動作のクラスを呼び出します。

以降、他の状態については、その状態に操作できるキーが押されて時は、その動作ができるメソッドを呼び出して処理できるようにしています。

\lstinputlisting[
  caption={GamePanel.java(一部抜粋)},
  firstline=487,
  lastline=501
]{src/view/GamePanel.java}
このクラスは、keyPressedで行われた変更を元の状態に戻します。このようにすることで、毎フレーム押された状態だけを処理することができます。

\subsubsection{第二ボスZeusの実装}
第二ボスZeusの実装は、Zeus本体とZeusが攻撃する雷(Lighting)の二つのファイルを作成して実装しました。
まず、本体についてです。
\lstinputlisting[
  caption={Zeus.java(一部抜粋)},
  firstline=8,
  lastline=8
]{src/model/Zeus.java}
\lstinputlisting[
    caption={Zeus.java (一部抜粋)},
    firstline=43,
    lastline=111
]{src/model/Zeus.java}
このZeus.javaは、Bossを継承したクラスです。
通常攻撃としては、2種類あります。それをablity1,ablity2として表しています。そして、このボスは、第一段階と第二段階(secondPhase)の二つの状態を持っています。
moveメソッドでは、ablity2を発動していない時は、画面の端に行ったら折り返しながら、攻撃をするという動きをしています。そして、ablity2は、画面の端から端まで素早く動きながら攻撃し続けます。この端から恥まで動く動きをramdomを用いてランダムに決めています。
このmoveメソッドの工夫点としては、変数isSecondPhaseとability2Startedの用意である。ablity2が発動できる状態になったら、ability2Startedをtrueにし、そうでなければfalseにしておくことで、簡単にZeusがablity1を使うのか、ablity2を使うのかを判別することができる。また、同様に、isSecondPhaseにより、Zeusの第二段階にいるかをこれによって判別することができることで、それぞれの段階の攻撃を用意することが容易にできた。

次に、雷(Lighting)についてです。
\lstinputlisting[
    caption={Lighting.java (一部抜粋)},
    firstline=7,
    lastline=55
]{src/model/Lighting.java}
Lightingは、Zeusの攻撃手段であり、Zeusを倒すと、Playerの特殊能力にもなります。そのため、friendlyという変数を用意することで、同じLightingでも
Playerの攻撃として用いたり、Bossの攻撃としても用いることができます。この変数を追加することで、Lightingを味方用と、ボス用と二つファイルを作らずに、
一つのファイルで完結させることができました。また、変数isSecondPhaseにより、第二段階に入ったことが判別できる。第二段階に入ったときに、画像を変えることでLightingでもより強く見せることができました。
このような変化を加えることで、ゲーム性が増しました。

{\raggedleft (文責:齋藤)\par}

\subsection{佐々木担当}
私は主にView層におけるUIの実装と、Model層におけるプレイヤーのライフ・ダメージ処理を担当しました。\\
\textbf{1、ゲーム画面およびHUDの実装}\\
\indent ユーザーが直感的に状況を把握できるよう、ゲームの状態に応じた画面遷移を実装しました。具体的には、ゲーム起動時の「タイトル画面」、操作説明を含む「スタート画面」、そして「ゲームオーバー画面」を作成しました。ゲームオーバー時には最終スコアを表示し、キー入力(Cで継続、Qで終了)によって次のアクションを選択できる機能を設けています。また、プレイ画面上部のHUDでは、現在のスコア、進行中のステージ数、およびプレイヤーの残りライフをリアルタイムで表示するようにしました。ライフは数字ではなくハートのアイコンを用いることで、視認性を高めています。
\begin{lstlisting}[caption = HUDの描画(GamePanel.java)]
    public class GamePanel extends JPanel implements KeyListener {
        private void drawTopHUD(Graphics g) {
            // Draw dark background bar
            g.setColor(new Color(50, 50, 80));
            g.fillRect(0, 0, GameConstants.WINDOW_WIDTH, GameConstants.HUD_HEIGHT);

            // Draw white separator line
            g.setColor(Color.WHITE);
            g.drawLine(0, GameConstants.HUD_HEIGHT, GameConstants.WINDOW_WIDTH, GameConstants.HUD_HEIGHT);

            // Font settings
            setPixelFont(g, 18f);
            int textY = 35;

            // A. Score
            g.drawString("SCORE:" + model.getScore(), 10, textY);

            // B. Stage (Centered)
            String stageText = model.getStageText();
            int stageX = (GameConstants.WINDOW_WIDTH - g.getFontMetrics().stringWidth(stageText)) / 2;
            g.drawString(stageText, stageX, textY);

            // C. Hearts / Lives (Right aligned)
            int maxLives = GameConstants.PLAYER_MAX_LIVES;
            int currentLives = model.getLives();
            int heartSize = 32;
            int spacing = 8;
            int startX = GameConstants.WINDOW_WIDTH - 20 - (maxLives * (heartSize + spacing));
            int heartY = (GameConstants.HUD_HEIGHT - heartSize) / 2;

            for (int i = 0; i < maxLives; i++) {
                // Determine which icon to draw (Full or Empty)
                BufferedImage icon = (i < currentLives) ? ResourceManager.heartFullImg : ResourceManager.heartEmptyImg;

                if (icon != null) {
                    g.drawImage(icon, startX + (i * (heartSize + spacing)), heartY, heartSize, heartSize, null);
                } else {
                    // Fallback drawing if images are missing
                    g.setColor(i < currentLives ? Color.RED : Color.GRAY);
                    g.fillOval(startX + (i * (heartSize + spacing)), heartY, heartSize, heartSize);
                }
            }
        }
    }
\end{lstlisting}

\noindent \textbf{2、ライフ制度と無敵時間の導入}\\
\indent ゲームの難易度調整とプレイ体験の向上のため、プレイヤーに3つのライフを付与するシステムを構築しました。初期の実装では、敵や弾に接触した際、当たり判定が連続して発生しライフが一瞬で0になってしまう「多段ヒット」の問題が発生していました。これを解消するために、ダメージを受けた直後に一定時間の「無敵時間(クールタイム)」を導入しました。ダメージ発生時に damageTimer をセットし、このタイマーが作動している間は新たなダメージを受け付けないように処理を分岐させています。さらに、無敵時間中はプレイヤーキャラクターを点滅させることで、視覚的にもダメージを受けたことと無敵状態であることをユーザーが理解しやすいようにしました。
\begin{lstlisting}[caption = ライフとタイマーの定義・初期化(GameModel.java)]
    public class GameModel {
        // 追加:ライフ機能用変数
        private int lives; // 初期ライフ
        private int damageTimer; // ダメージを受けた後の無敵時間(フレーム数)

        public void initGame() {
            // 追加:ライフ初期化
            lives = GameConstants.PLAYER_MAX_LIVES;
            damageTimer = 0;
        }
    }
\end{lstlisting}
\begin{lstlisting}[caption = 無敵時間のカウントダウン処理(GameModel.java)]
    public class GameModel {
        public void update() {
            // 無敵時間の更新
            if (damageTimer > 0) {
                damageTimer--;
            }
        }

        // ダメージ処理メソッド
        private void playerTakesDamage() {
            if (damageTimer == 0) { // 無敵時間中でなければダメージ
                lives--;
                damageTimer = 120; // 180フレーム(約3秒)無敵にする
                System.out.println("Damage taken! Lives remaining: " + lives);

                if (lives <= 0) {
                    state = GameState.GAMEOVER;
                    System.out.println("GAME OVER");
                }
            }
        }
    }
\end{lstlisting}
{\raggedleft (文責:佐々木)\par}

\subsection{Puller担当}

\section{実行例}
% 実行例をスクリーンショットを使って,分かりやすく説明してください.特に,作成したプログラムの特徴となるような機能については,実際の使用例を分かりやすく説明するようにしてください.
コードを実行すると、左の画面が出てくる。スペースキーを押すと右の画面に移る。
\begin{figure}[H]
    \centering
    \begin{minipage}{0.3\columnwidth}
        \centering
        \includegraphics[width=\textwidth]{title.eps}
        \caption{タイトル画面}
    \end{minipage}
    \hfill
    \begin{minipage}{0.3\columnwidth}
        \centering
        \includegraphics[width=\textwidth]{screenshot/aml1h-qx0nm.eps}
        \caption{スタート画面}
    \end{minipage}
\end{figure}
再びスペースキーを押すと、右上の画面から左下の画面に移る。この画面になったら、方向キーとスペースキーを使って敵を倒す。一定の敵を倒すと、右の画面に移り、第一ボスのアポロンが登場する。
\begin{figure}[H]
    \centering
    \begin{minipage}{0.3\columnwidth}
        \centering
        \includegraphics[width=\textwidth]{screenshot/screen_2026_02_1853358.eps}
        \caption{第一ステージ}
    \end{minipage}
    \hfill
    \begin{minipage}{0.3\columnwidth}
        \centering
        \includegraphics[width=\textwidth]{screenshot/screen_2026_02_1853419.eps}
        \caption{第一ボス(アポロン)登場}
    \end{minipage}
\end{figure}
右上の画面から、スペースキーを押すと、左下にある第一フェーズの第一ボスアポロンと戦う。攻撃し続けると、右下の画面のように変化する。
\begin{figure}[H]
    \centering
    \begin{minipage}{0.3\columnwidth}
        \centering
        \includegraphics[width=\textwidth]{screenshot/screen_2026_02_1853424.eps}
        \caption{第一ボス(アポロン) 第一フェーズ}
    \end{minipage}
    \hfill
    \begin{minipage}{0.3\columnwidth}
        \centering
        \includegraphics[width=\textwidth]{screenshot/screen_2026_02_1853622.eps}
        \caption{第一ボス(アポロン) 第二フェーズ}
    \end{minipage}
\end{figure}
第一ボスアポロンを倒すと、左下の画面に移る。その画面で、スペースキーを押すと第二ステージが始まる。
\begin{figure}[H]
    \centering
    \begin{minipage}{0.3\columnwidth}
        \centering
        \includegraphics[width=\textwidth]{screenshot/screen_2026_02_1853430.eps}
        \caption{第一ステージ クリア}
    \end{minipage}
    \hfill
    \begin{minipage}{0.3\columnwidth}
        \centering
        \includegraphics[width=\textwidth]{screenshot/screen_2026_02_1853441.eps}
        \caption{第二ボス(ゼウス) 登場}
    \end{minipage}
\end{figure}
左下のように、第一ボスを倒して、得られた太陽を使いながら第二ステージを攻略する。ステージが進むと、右下のように第二ボスゼウスが登場する。
\begin{figure}[H]
    \centering
    \begin{minipage}{0.3\columnwidth}
        \centering
        \includegraphics[width=\textwidth]{screenshot/aanjk-xj71v.eps}
        \caption{第二ステージ 特殊能力1(太陽)}
    \end{minipage}
    \hfill
    \begin{minipage}{0.3\columnwidth}
        \centering
        \includegraphics[width=\textwidth]{screenshot/a8sri-5dqx7.eps}
        \caption{第二ステージ}
    \end{minipage}
\end{figure}
右上の画面でスペースキーを押すと、左下のように、第二ボスと戦う。ダメージを与え続けると、右下のように変化する。
\begin{figure}[H]
    \centering
    \begin{minipage}{0.3\columnwidth}
        \centering
        \includegraphics[width=\textwidth]{screenshot/screen_2026_02_1853451.eps}
        \caption{第二ボス(ゼウス) 第一フェーズ}
    \end{minipage}
    \hfill
    \begin{minipage}{0.3\columnwidth}
        \centering
        \includegraphics[width=\textwidth]{screenshot/screen_2026_02_1853552.eps}
        \caption{第二ボス(ゼウス) 第二フェーズ}
    \end{minipage}
\end{figure}
第二ボスを倒すと、左下の画面に移る。スペースキーを押すと第三ステージが始まる。
\begin{figure}[H]
    \centering
    \begin{minipage}{0.3\columnwidth}
        \centering
        \includegraphics[width=\textwidth]{screenshot/a4417-wlsd5.eps}
        \caption{第二ステージ クリア}
    \end{minipage}
    \hfill
    \begin{minipage}{0.3\columnwidth}
        \centering
        \includegraphics[width=\textwidth]{screenshot/aer32-13xux.eps}
        \caption{第三ステージ 特殊能力1(太陽)}
    \end{minipage}
\end{figure}
右上、左下のようにボスを倒して得られた特殊能力を使ってスコアを稼ぐ。残りライフが0になると、右下のようにゲームオーバーになる。
\begin{figure}[H]
    \centering
    \begin{minipage}{0.3\columnwidth}
        \centering
        \includegraphics[width=\textwidth]{screenshot/aki7j-iqzs4.eps}
        \caption{第三ステージ 特殊能力2(雷)}
    \end{minipage}
    \hfill
    \begin{minipage}{0.3\columnwidth}
        \centering
        \includegraphics[width=\textwidth]{screenshot/ad1up-zu78p.eps}
        \caption{ゲームオーバー画面}
    \end{minipage}
\end{figure}


{\raggedleft (文責:齋藤)\par}
\section{考察}
% 完成したプログラムに対する考察,コメントを述べてください.当初予定していた通りの物ができたか考察してください.また,考察を踏まえて,今後の改良点についても述べてください.

\section{感想}
% グループでの作業を通しての反省や感想,それから考察できること.
% 今後の各自の担当部分の課題.やり残したこと.
% Javaやオブジェクト指向,MVCモデルなど学習内容に関する感想。
% 「プログラミング演習」の授業に関する感想や要望.

\subsection{齋藤}
この授業で、初めてJavaについて学んだ。Javaは、基本的な構文は、C言語に似ていたがオブジェクト指向型プログラミングを初めて習ったので難しく感じた。しかし、グループ内課題でプログラミングをやっていると、オブジェクト指向の便利さに気づいた。例えば、継承である。継承は、親クラスの内容を引き継ぐ子クラスに行うものであり、それが今回のシューティングゲームを開発している時に、重要となった。なぜなら、
このゲーム開発では、player, Boss, Arrow, 雑魚敵など様々なキャラクターやオブジェクトが登場してきて、それらを全て一から書くのが大変だから、GameObjectと全体に共通するものを書いて大きなグループを作り、そこから派生して、Boss、BossProjectileなど細かいグループを作ってコードを作成できることが、整理しやすくそれが作成のしやすさに繋がっていたからである。

私は、MVCモデルのCとMの一部を担当しました。Cでは、Playerの上下左右の移動や、矢を放ったり、特殊能力を放つなどのさまざま操作ができるようにしました。Playerの移動の際に、1フレーム毎に縦、横の4方向しか移動が出来ないので、スムーズに斜めに移動することができるようにするなど、移動方向を増やしてみたいと思った。

この授業を通して、Javaの基本的なことから実践まで学べるようになっていることが、他の授業には良い点であると思った。実践的な練習することで、プログラミングの理解が広がるだけではなく、周りとコミュニケーションをとり、進捗を確認し合いながら進めることは、この授業で初めて行ったことであり、今後に活かすことができる経験であった。


{\raggedleft (文責:齋藤)\par}
\subsection{佐々木}
本授業を通じて、初めてのJavaの学習から始まり、最終的にはグループでのゲーム開発という実践的な演習に取り組みました。Java特有のオブジェクト指向における「継承」の概念は、初めは難しかったものの、開発が進むにつれてコードの再利用性を高めるために不可欠な技術であることを実感しました。\\
\indent 開発手法としてはMVCモデルを採用し、私は主にVとMの一部を担当しました。具体的には、ゲームの入り口となるタイトル画面やスタート画面、そしてプレイ画面やゲームオーバー画面といったUI全般の実装に注力しました。単にグラフィックを表示するだけでなく、ゲームの状態に応じて動的に表示を切り替えるロジック部分の構築も経験し、UIと内部処理がどう連携すべきかを深く意識することができました。\\
\indent また、GitHubを用いた共同開発も初めての経験でした。ブランチの管理やコンフリクトの解消など、個人開発では味わえない困難もありましたが、チームで一つのソースコードを育て上げる過程でバージョン管理の重要性を学びました。技術的な習得だけでなく、メンバーと協力して一つのプロジェクトを完遂する難しさと喜びを知ることができ、非常に実りある経験となりました。

{\raggedleft (文責:佐々木)\par}

\subsection{Puller}

\newpage
\section{付録1:操作法マニュアル}
% そのプログラムを初めて使う人向けの説明書を書いてください.ドローエディタなら操作法の説明,ゲームならキー操作の説明などを書いてください.スクリーンキャプチャした画像に説明を書き込んだりしてもいいでしょう。2ページ程度の簡潔な説明書で構いません。

ゲームを起動すると、タイトル画面が出てきます。Spaceキーを押してスタート画面に移り、再度Spaceキーを押すことでゲームが始まります。
\begin{figure}[H]
    \centering
    \begin{minipage}{0.4\columnwidth}
        \centering
        \includegraphics[width=\textwidth]{title.eps}
        \caption{タイトル画面}
    \end{minipage}
    \hfill
    \begin{minipage}{0.4\columnwidth}
        \centering
        \includegraphics[width=\textwidth]{start.eps}
        \caption{スタート画面}
    \end{minipage}
\end{figure}

次に、プレイ中のキー操作について説明します。
\begin{table}[H]
    \centering
    \caption{プレイ中のキー操作}
    \begin{tabular}{|c|c|m{20em}|}
        \hline
        移動 & カーソルキー & 画面全体を縦横無尽に移動。 \\
        \hline
        攻撃 & Spaceキー & 長押しで連射が可能。 \\
        \hline
        一時停止、再開 & P & プレイ中に押すことでゲームを一時停止、再度押すことでゲームを再開。 \\
        \hline
        特殊スキル & 1,2,3 & 各ステージのボスを倒すことで手に入る特殊スキルは、それぞれ1,2,3を押すことで発動。 \\
        \hline
    \end{tabular}
\end{table}

ゲームオーバー時には、Cを押すとゲームをもう一度プレイ(Continue)でき、Qを押すとゲームをやめる(Quit)ことができます。
\begin{figure}[H]
    \centering
    \includegraphics[scale=0.25]{gameover.eps}
    \caption{ゲームオーバー画面}
\end{figure}

{\raggedleft (文責:佐々木)\par}

\newpage
\section{付録2:プログラムリスト}
% プログラムリストは本文中には説明に必要な部分だけ抜粋して掲載して,プログラムリスト全体はレポートの最後に「付録」として付ける。プログラムコードには,主なメソッド,フィールドの説明など最低限のコメントは付ける。さらに,cat -n コマンドなどで行番号を付与してから載せること.

\lstinputlisting[caption = Main.java]{src/main/Main.java}

\lstinputlisting[caption = GameModel.java]{src/model/GameModel.java}
\lstinputlisting[caption = GameConstants.java]{src/model/GameConstants.java}
\lstinputlisting[caption = GameObject.java]{src/model/GameObject.java}
\lstinputlisting[caption = GameState.java]{src/model/GameState.java}
\lstinputlisting[caption = Alignment.java]{src/model/Alignment.java}
\lstinputlisting[caption = Background.java]{src/model/Background.java}
\lstinputlisting[caption = Projectile.java]{src/model/Projectile.java}
\lstinputlisting[caption = Player.java]{src/model/Player.java}
\lstinputlisting[caption = Arrow.java]{src/model/Arrow.java}
\lstinputlisting[caption = BossProjectile.java]{src/model/BossProjectile.java}
\lstinputlisting[caption = Boss.java]{src/model/Boss.java}
\lstinputlisting[caption = Apollo.java]{src/model/Apollo.java}
\lstinputlisting[caption = Sun.java]{src/model/Sun.java}
\lstinputlisting[caption = Zeus.java]{src/model/Zeus.java}
\lstinputlisting[caption = Lighting.java]{src/model/Lighting.java}
\lstinputlisting[caption = HostileEntity.java]{src/model/HostileEntity.java}
\lstinputlisting[caption = EnemySpawner.java]{src/model/EnemySpawner.java}
\lstinputlisting[caption = Boulder.java]{src/model/Boulder.java}
\lstinputlisting[caption = Cyclops.java]{src/model/Cyclops.java}
\lstinputlisting[caption = Feather.java]{src/model/Feather.java}
\lstinputlisting[caption = Harpy.java]{src/model/Harpy.java}
\lstinputlisting[caption = Minion.java]{src/model/Minion.java}

\lstinputlisting[caption = GamePanel.java]{src/view/GamePanel.java}
\lstinputlisting[caption = ResourceManager.java]{src/view/ResourceManager.java}

{\raggedleft (文責:佐々木)\par}

\end{document}