\documentclass[12pt,a4paper]{jsarticle}

\usepackage[utf8]{inputenc}
\usepackage[T1]{fontenc}
\usepackage[dvipdfmx]{graphicx}
\usepackage{lmodern}
\usepackage{amsmath,amssymb}
\usepackage{hyperref, url}
\usepackage{caption}
\usepackage{float}
\usepackage{here}
\usepackage{array}

\title{情報理工学域 メディア情報学プログラム\\
    プログラミング演習 最終レポート}
\author{DX2 グループ11\\
    2411593 齋藤寛仁\\
    2411603 佐々木晃誠\\
    2411712 Puller Stefano Daniele}
\date{2026年2月18日}

\begin{document}
\maketitle

\newpage
\section{概要説明}
% どの様な目的のプログラムを実現したか,どのような機能を実装したか,をまず最初に説明してください.
% 必要に応じてスクリーンキャプチャした画像も張り付けてください.
% 作業の分担など,作業をどのように進めたかも説明して下さい.

私たちのグループは、シンプルなシューティングゲームを作成しました。画面上部から敵がランダムに降りてきて攻撃します。敵を躱したり、矢を放って敵を迎え撃ちます。敵を倒すことでスコアが上昇し、一定のスコアに達するとボスが登場します。各ステージのボスを倒すことで、次のステージに移動します。ボスを倒すとそのキャラの特殊スキルを自分の能力として使うことができるようになります。プレイヤーのライフが0になるとゲームオーバーです。\\
\indent 基本的なキー操作は移動と攻撃で、カーソルキーで移動し、Spaceキーで矢を放って攻撃します。プレイ中に一時停止することも可能です。\\
\indent プレイヤーのライフは3が上限で、敵に当たるか敵の攻撃を受けると1減ります。ボスが登場したタイミングと、ボスを倒して次のステージに進んだタイミングでライフが全回復します。\\
\indent プレイ画面の上部には、現在のスコアとステージ数、プレイヤーの残りライフが表示されています。特殊スキルは画面下部に順次追加されます。
\begin{figure}[H]
    \centering
    \includegraphics[scale=0.25]{play.eps}
    \caption{プレイ画面}
\end{figure}
作業分担は、

\section{設計方針}
% プログラムの基本構想を述べて下さい.
% どのようなアルゴリズム,データ構造を用いて,目的の処理を実現するのか説明しましょう.なぜそのようなアルゴリズム,データ構造を用いたかという考察も含めてください.
% ただし,ここでは,プログラムコードの細部には触れる必要はありません.それは次の章で行います.
% 今回はJavaプログラミングなので,どのようなクラスを用意して,どのように利用するかを説明してください。クラス図を使用して,主要なクラス間の関係を説明するしましょう。
% なお,クラス図を書くときは,すでにあるものを利用したクラス(レポートで特に説明しないクラス)と,新しく作成したクラスは区別が付くようにしましょう.

\section{プログラムの説明}
% 作成したすべてのクラスを説明してください.主な各クラスの主なメソッド,フィールドの説明を行ってください.必要に応じてプログラムリストを抜粋して説明しましょう.
% ここは, 必ず,メンバ全員がそれぞれ自分の担当した部分についての説明を書くようにしてください.発表会でのプレゼンと同様に 各自の担当部分のセールスポイト(実装した機能,採用したデータ構造やアルゴリズムなどについて) も書いて下さい.「文責:○○」も忘れずに書くようにしてください.
% 本授業HPにあるひな形コード,もしくはWeb上のコードをひな形として利用した場合は,その旨明記して下さい。Web上のコードを元にした場合は,必ずURLも書いてください。その場合,ひな型に最初から含まれていた,主要なメソッド,フィールドについても説明を行って,※などの印を付けて,追加したメソッド,フィールドと区別できるようにしてください。

\subsection{齋藤担当}

\subsection{佐々木担当}

\subsection{Puller担当}

\section{実行例}
% 実行例をスクリーンショットを使って,分かりやすく説明してください.特に,作成したプログラムの特徴となるような機能に ついては,実際の使用例を分かりやすく説明するようにしてください.

\section{考察}
% 完成したプログラムに対する考察,コメントを述べてください.当初予定していた通りの物ができたか考察してください.また,考察を踏まえて,今後の改良点についても述べてください.

\section{感想}
% グループでの作業を通しての反省や感想,それから考察できること.
% 今後の各自の担当部分の課題.やり残したこと.
% Javaやオブジェクト指向,MVCモデルなど学習内容に関する感想。
% 「プログラミング演習」の授業に関する感想や要望.

\section{付録1:操作法マニュアル}
% そのプログラムを初めて使う人向けの説明書を書いてください.ドローエディタなら操作法の説明,ゲームならキー操作の説明などを書いてください.スクリーンキャプチャした画像に説明を書き込んだりしてもいいでしょう。2ページ程度の簡潔な説明書で構いません。

ゲームを起動すると、タイトル画面が出てきます。Spaceキーを押してスタート画面に移り、再度Spaceキーを押すことでゲームが始まります。
\begin{figure}[H]
  \centering
  \begin{minipage}{0.4\columnwidth}
    \centering
    \includegraphics[width=\textwidth]{title.eps}
    \caption{タイトル画面}
  \end{minipage}
  \hfill
  \begin{minipage}{0.4\columnwidth}
    \centering
    \includegraphics[width=\textwidth]{start.eps}
    \caption{スタート画面}
  \end{minipage}
\end{figure}

次に、プレイ中のキー操作について説明します。
\begin{table}[H]
    \centering
    \caption{プレイ中のキー操作}
    \begin{tabular}{|c|c|m{20em}|}
        \hline
        移動 & カーソルキー & 画面全体を縦横無尽に移動。 \\
        \hline
        攻撃 & Spaceキー & 長押しで連射が可能。 \\
        \hline
        一時停止、再開 & P & プレイ中に押すことでゲームを一時停止、再度押すことでゲームを再開。 \\
        \hline
        特殊スキル & 1,2,3 & 各ステージのボスを倒すことで手に入る特殊スキルは、それぞれ1,2,3を押すことで発動。 \\
        \hline
    \end{tabular}
\end{table}

ゲームオーバー時には、Cを押すとゲームをもう一度プレイ(Continue)でき、Qを押すとゲームをやめる(Quit)ことができます。
\begin{figure}[H]
    \centering
    \includegraphics[scale=0.25]{gameover.eps}
    \caption{ゲームオーバー画面}
\end{figure}

{\raggedleft (文責:佐々木)\par}

\section{付録2:プログラムリスト}
% プログラムリストは本文中には説明に必要な部分だけ抜粋して掲載して,プログラムリスト全体はレポートの最後に「付録」として付ける。プログラムコードには,主なメソッド,フィールドの説明など最低限のコメントは付ける。さらに,cat -n コマンドなどで行番号を付与してから載せること.

\end{document}